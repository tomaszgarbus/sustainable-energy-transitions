\section{Complex Societal Problems}

Dorien J. DeTombe, 2002

\subsection{Introduction}

\subsubsection{The problem handling process}

Each phase of the process should be carefully reflected, discussed and guided.

The first subcycle: \textbf{defining the problem}:

\begin{itemize}
	\item becoming aware of the problem and building mental image
	\item extending the mental idea by hearing, talking, thinking, reading
	about the problem
	\item putting the problem on the agenda and deciding to handle it
	\item forming the problem handling team and starting to analyze it
	\item gathering data, exchanging knowledge, and forming hypotheses
	about the problem
	\item formulating the conceptual model of the problem
\end{itemize}

The second subcycle: \textbf{changing the problem}:

\begin{itemize}
	\item constructing the empirical model and the desired goal
	\item defining the handling space
	\item constructing and evaluating scenarios
	\item suggesting interventions
	\item implementing interventions
	\item evaluating interventions
\end{itemize}

\subsection{What are complex societal problems?}

Examples:

\begin{itemize}
	\item transportation problems
	\item environmental problems
	\item urban planning problems
	\item healthcare problems
	\item water problems
	\item Internet problems
	\item they are seldom really solved
\end{itemize}

Characteristics of CSP:

\begin{itemize}
	\item uncertain what the problem looks like
	\item limited knowledge and data
	\item governmental or continental scale
	\item many cultures involved with different traditions of problem
	handling
	\item group of actors involved changes during the process
	\item power and emotions of different actors
	\item often caused by greed, war, corruption, indifference and/or huge
	private profit
	\item political agendas at play
\end{itemize}

\textit{There is also a tendency to privatize the
benefits of the problems and to socialize the costs.}

\subsection{The COMPRAM method}
DeTombe, 1994

Summary:

\begin{itemize}
	\item democratic, transparent and structured
	\item problems are solved cooperatively by diverse teams under the
	problem owner and coordinator
	\item the actors should be involved in the problem solving from the
	very beginning
	\item the process should take 6-12 months
	\item interventions, in order to be effective, have to be integrated
\end{itemize}

\subsubsection{The 6 steps of COMPRAM}

\begin{itemize}
	\item analyze and describe the problem by a team of neutral content
	experts
	\item analyze and describe the problem by different teams of actors
	\item find interventions by experts and actors together
	\item anticipate the societal reactions
	\item implement the interventions
	\item evaluate the changes
\end{itemize}

Handling CSP requires 3 elements: knowledge, power and emotions.

\subsubsection{The seven-layer model}

It is meant to help integrate and understand knowledge from various
disciplines.

Layers:

\begin{itemize}
	\item normal verbal language layers:
	\begin{itemize}
		\item description of the problem
		\item definition concepts and phenomena
		\item theories hypotheses assumptions experience intuition
	\end{itemize}
	\item visual language of the mental model
	\begin{itemize}
		\item knowledge islands
		\item semantic model
		\item casual model
	\end{itemize}
	\item mathematical language of the simulation model
	\begin{itemize}
		\item system dynamic simulation model
	\end{itemize}
\end{itemize}

The framework avoids the term "solving" the problem because CSP are rarely
really solved. Instead it talks of "changing" the problem.
