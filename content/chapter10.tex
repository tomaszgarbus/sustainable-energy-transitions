\section{Chapter 10: Sustainable And Just Energy Strategies}

Agriculture contributes around 25\% GHG emissions, out of which 80\% is from
animal agriculture.

Livestock production accounts for 70\% of agricultural land and 30\% of total
land.

Another 25\% GHG emissions are from electricity and heat.

The \textbf{food-energy-water nexus} is a phrase used to describe the
biophysical, natural, social, and behavioral processes that are interconnected
by food, energy, and water.

FEW systems can be physical, natural, biological, or social and behavioral
processes.

Photovoltaic (PV) systems displace
thermoelectric power generation, which consumes roughly 39\% of freshwater
withdrawals in the US.

Freshwater use for producing
food and fiber constitutes roughly 70\% of the freshwater used in agriculture.

Global meat production rose to 350 million tons in 2014 from
78 million tons in 1961.

37\% of global methane is from cattle.

Methane has 23-25 times heat trapping ability of that of CO$_2$.

Livestock also contribute 64\% of ammonia emissions, which contribute to acid
rain and acidification of land and waterways. Cattle and livestock are
water-intensive also, requiring over 900 liters of water to produce a gallon
of milk, and while 1.3 billion tons of grains are consumed by farm animals
each year, nearly all of it is fed to cattle.

Extended producer respon-
sibility (EPR) is one means to lessen the impacts of end-of-life (EOL)
management problems with products after they outlive their useful life.

\subsection{Techno-ecological synergies}

Example: agrovoltaics: rice grown under a canopy of photovoltaics

Example types of solar techno-ecological synergies:

\begin{itemize}
	\item installations over previously disturbed land
	\item over water
	\item as distributed energy generators
	\item in agroecological systems
	\item distributed throughout the electricity grid
\end{itemize}

\textbf{Floatovoltaics} are photovoltaics integrated with materials that allow
them to float and be installed in lakes, reservoirs, wastewater treatment
ponds, ocean bays, and other marine waterways.

Benefits of agrivoltaics:

\begin{itemize}
	\item managed and native pollinators
	\item increased water use efficiency
	\item soil erosion prevention
	\item can alter microclimatic conditions to keep the PV systems cooler
	and make them operate more efficiently
\end{itemize}

Benefits of floatovoltaics:

\begin{itemize}
	\item reducing algae growth
	\item preventing water loss from evaporation
\end{itemize}

Benefits of built-up systems (PV on rooftops):

\begin{itemize}
	\item insulating buildings
	\item cooling buildings
	\item energy savings
\end{itemize}

\subsection{Moving forward on an energy transition towards decarbonization}


