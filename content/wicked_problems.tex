\section{Wicked energy transition?}
Johann Köppel, Sustainable Energy Transition

What characterizes wicked problems:

\begin{itemize}
	\item There is no definitive formulation. Definition of the problem is
	part of the problem.
	\item No stopping rule -- actors stop their actions at their
	discretion.
	\item Resolutions are not true-or-false but good-or-bad (or
	better-or-worse).
	\item There is no immediate and ultimate test for a solution of the
	problem.
	\item Every resolution of a wicked problem is one-shot operation;
	there is no opportunity to learn by trial-and-error, every attempt
	counts significantly.
	\item Don't have an exhaustively describable set of potential
	solutions, not a set of well-defined operations to be incorporated.
	\item They are essentially unique.
	\item Every wicked problem is a symptom of another problem.
	\item The existence of a wicked problem can be explained in multiple
	ways. The choice of explanation determines the nature of the solution.
	\item The planner has no right to be wrong. Planners are liable for the
	consequences of their actions.
\end{itemize}
