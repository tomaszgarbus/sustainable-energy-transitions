\section{Chapter 4: Fossil Fuels}

We can think of useful energy in the forms of stocks and flows. Stocks of
energy and natural resources can be drawn down, while flows of energy and
natural resources are constantly replenishing.

A total of 90\% of air pollution -- nitrogen oxides, sulfur oxides, particulate
matter, heavy metals like lead, and volatile organic compounds -- is caused by
combustion according to US EPA.

Ozone in the upper reaches of the atmosphere blocks otherwis harmful
ultra-violet light. But at ground level it is a lung irritant and can cause
significant heart and lung damage.

Chemistry of the breakdown of nitrogen oxides to ozone:
\begin{itemize}
	\item \textbf{Nitric oxide, primary pollutant}:
	N$_2$ $+$ O$_2$ $+$ high temperatures $\rightarrow$ 2NO
	\item \textbf{Nitrogen dioxide}:
	2 NO $+$ O$_2$ $\rightarrow$ 2 NO$_2$
	\item \textbf{NO$_x$ + oxygen radical}:
	NO$_2$ + higher energy sunlight $\rightarrow$ NO $+$ O
	\item \textbf{Ground-level ozone}:
	O $+$ O$_2$ $\rightarrow$ O$_3$
\end{itemize}

\subsection{Coal}

Creation of coal:
\begin{itemize}
	\item processes started during carboniferous period 360 to 286 Myr
	\item vegetation and other sources of carbon are sedimented upon,
	starved of oxygen, and decay under pressure, at high heat, and over
	time, to form coal
	\item from cellulose and lignin in plants
	\item youngest form is \textbf{peat}, produced from compression of
	sedimentary layers of rock, soil and sand
	\item brown coal forms from peat given more time, heat and pressure
	\item most brown coal comes from Quaternary period starting 2 Myr ago
\end{itemize}

\textbf{Energy Density} -- a metric that represents a quantity of energy per
unit area of unit volume.

Coal mining: 
\begin{itemize}
	\item 7.3 Bt globally per year
	\item China 2300 Mt
	\item India 708 Mt
	\item US 672 Mt
	\item Australia 503 Mt
	\item Indonesia 461 Mt
\end{itemize}

Over 90\% of coal is used in the electricity sector.

China's plans imply that it will have consumed about 50\% of global coal ever
produced.

3 of the most fatal coal mine disasters were caused by firedamp -- an explosive
mix of methane (CH$_4$), coal dust, and hydrogen sulfide (H$_2$S).

In total, 60\% of PM is from coal, 45\% of SO$_2$, 30\% of NO$_x$ and 80\%
of mercury.

Mercury is a contaminant found in coal and after combustion is deposited in the
environment.

\textbf{Carbon capture and storage (CCS)}: many CCS technologies require
converting coal into syngas -- synthetic gas -- that comprises hydrogen,
carbon monoxide and carbon dioxide. The reaction to produce syngas is:

Coal(mostly carbon, hydrogen) $+$ O$_2$ $+$ H$_2$O $\rightarrow$ H$_2$ $+$ CO

\subsection{Natural gas (fossil gas)}

Mainly methane, often referred to "dry" natural gas. "Wet" natural gas is the
collection of hydrocarbons extracted from the Earth, including water, ethane,
propane, butane, methane and impurities.

\subsection{Petroleum (crude oil)}

Final products: gasoline and diesel. The word \textit{petroleum} refers to the
primary energy resource extracted from Earth.

Despite strong climate action and renewable energy adoption, oil consumption is
expected to increase to 123 million barrels per day by the year 2025.

\textit{
One pivotal player in global oil production is the Organization of the
Petroleum Exporting Countries (OPEC). The cartel operates by ensuring that
prices and production outputs are set to achieve favorable outcomes for
members. Saudi Arabia is what oil economy experts call a swing producer. This
means they can swing production in short notice because of significant amount
of oil production capacity. But recent years have seen some diminishment in
the power of OPEC and the rise of the US and natural gas industries over
petroleum ones.
}

The 1969 Union Oil spill from a platform near Santa Barbara, California, is
considered to have catalyzed the US government into setting up the
Environmental Protection Agency (EPA) as well as passing Clean Air Act and
Clean Water Act.

\subsection{Tar sands, oil sands}

\begin{itemize}
	\item 75\% of tar sands are in Alberta, Canada
	\item combinations of sand, clay, water and bitumen
	\item bitumen is a hydrocarbon and can be used for any kind of oil or
	gasoline
	\item 14\% more GHG then conventional oil
\end{itemize}
