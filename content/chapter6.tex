\section{Chapter 6: Sustainable Energy Indicators}

\subsection{Industrial ecology}

\textbf{Industrial ecology} is a metaphor used to describe industrial systems
that do not create waste because processes use waste as inputs to other
industrial processes.

Commoner's four laws of ecology:
\begin{itemize}
	\item everything must go somewhere
	\item everything is interconnected
	\item nature knows best
	\item there are no free lunches in nature
\end{itemize}

\subsection{Sustainability indicators}

Corporate Social Responsibility (CSR) has been increasing (a large portion of
Fortune 500 companies report sustainability indicators to their shareholders
and the public).

Global Reporting Initiative (GRI)

Indicators:
\begin{itemize}
	\item air quality emissions
	\item water use and disposal
	\item greenhouse gases
	\item nitrogen effluents
	\item hundreds of others
\end{itemize}

\subsection{Carbon footprints}

A \textbf{carbon footprint} is a GHG inventory for an individual, product or
organization.

\textbf{ISO 14046} -- international standard for conducting carbon footprints.

The global range for personal carbon footprints varies from 1 to 100 metric
tons annually, and there are probably more severe high-use cases.

\subsection{Life-Cycle Assessment (LCA)}

\textit{
The Coca-Cola Company
set the standard for LCA in 1969 with a study to characterize the resource and
energy dimensions of glass versus aluminum containers. Eventually, this led to
European Commission issuing a Liquid Food Container Directive in 1985. The
International Standards Origination (ISO) 14000 has developed a standardized
LCA methodology.
}

\textbf{Attributional LCA} inventories all materials and assigns them
emissions.

\textbf{Consequential LCA} further asks what are the consequences of producing
that product or passing a policy.

\textit{
Arguably, the weakest aspect of using LCA in policy is the subjective elements
of the process throughout LCA—judgments about what to include, exclude, and
how to measure. But it is the interpretation phase that is often most
challenging because it relies on understanding the systems and how to put the
LCA findings into context.
}

\subsection{Energy Return On Investment (EROI)}

\subsection{Energy Payback Time (EPT)}

\textbf{EPT} is meant to illustrate how long the energy investments in a
renewable energy take to pay off the initial or life-cycle energy
investments.

The EPT for PV can range from six months as is claimed by some major
photovoltaic manufacturers like First Solar to two and three years for
crystalline silicon.

\textbf{GHG emissions} from the lifecycle of the PV system (g CO$_2$e):
$\text{GHGs} = W/(n \times pr \times I \times LT \times A)$
\begin{itemize}
	\item $I$ irridation (kWh per square meter per year)
	\item $n$: conversion efficiency
	\item $PR$: performance ratio
	\item $LT$: lifetime (years)
	\item $A$: area of the module (m$^2$)
\end{itemize}

%%%%%%%%%%%%%%%%%%%%%%%%%%%%%%%%%%%%%%%%%%%%%%%%%%%%%%%%%%%%%%%%%%%%%%%%%%%%%%%%

\subsection{Water footprints}

\textbf{Blue water footprint} estimates the volume of surface and groundwater
withdrawn, consumed or removed from the hydrologic cycle to produce a product.
It most accurately represents the amount of water used.

\textbf{Gray water footprint} is an indicator of water pollution, and it is
defined as the amount of water required to assimilate water pollution to an
acceptable threshold or standard.

$\text{Dilution factor} = \frac{\text{Contamination concentration}}
{\text{Water quality threshold}}$

$\text{Grey water footprint} = \frac{\text{Water disposed} \times
\text{dilution factor}}{\text{Natural gas production}}$

\textbf{Green water footprint} is the evaporative flow from the land surface.
Green water footprints apply mostly to products produced in agriculture or
forestry.

