\section{Chapter 3: Energy and the Social Sciences}

\subsection{Environmental Justice}

The term came into widespread use in the 1980s to describe the uneven
distribution of environmental burdens and public health harms.

Locations of incinerators, toxic waste disposal facilities, regional "cancer
alleys" and an increased exposure to air pollutants are disproportionately
found in low-income or minority communities.

\textbf{Environmental justice} -- the idea that no social group, communities,
or individuals should bear disproportional environmental or pollution burdens.

\subsection{Energy poverty}

\textbf{Energy poverty} is defined as not having access to modern energy
sources, or not being able to pay for those energy-related expenses.

The regions of the world with highest levels of energy poverty are:
\begin{itemize}
	\item sub-Saharan Africa
	\item Latin America
	\item China
	\item India
\end{itemize}

Globally, 1 of 5 people in the world have no access to electricity.

In Africa, excluding Egypt and South Africa, 1 of 5 people DO have access to
electricity.

China has seen the most rapid movement out of energy poverty due to great
success with rural electrification.

Indicators of energy poverty:
\begin{itemize}
	\item whether there is access to electricity
	\item type of cooking fuels, for example relying on wood, charcoal
	and/or dung for cooking constitutes poverty
\end{itemize}

IEA's definition of energy poverty: \textit{inability to cook with modern
cooking fuels and the lack of a bare minimum of electric lighting to read or
for other household and productive activities at sunset}

Other definitions of basic energy needs:
\begin{itemize}
	\item Swiss 2000-watt society: electricity consumption of 50-100 kWh
	per person per year
	\item UN Advisory Group on Energy and Climate Change: 50-100 kg of oil
	equivalent or modern fuel per person per year
\end{itemize}

Energy comes in different qualities, electricity is considered the highest
quality.

Inefficient burning of solid fuels on an open fire or traditional stove indoors
creates a dangerous cocktail of pollutants (CO$_2$, small particles, nitrogen
oxides, benzene, butadiene, formaldehyde, polyaromatic hydrocarbons and
others).

\subsection{Resource curse}

Also referred to as Dutch disease.

\textbf{Resource curse} -- the apparent contradiction that some communities
become over-reliant on an extractive natural resource and that social systems
are unable to ensure the community benefits from these riches.

Nigeria has huge petroleum wealth that is lost through circuits of poor policy
decisions, corruption, and inequality.

\subsection{Behavior and energy}

7\% of energy use could be reduced from behavioral changes alone.

Studies were made advocating in two groups for energy savings. One group
received information about efficiency and their personal savings, while the
other got information about the collective good of reduced air pollution. The
second one had a stronger behavioral response.

\textbf{Neighboorhood effect} for the adoption of rooftop residiential solar
energy.

\textbf{Jevons paradox}: \textit{
If the quantity of coal used in a blast-furnace, for instance, be diminished
in comparison with the yield, the profits of the trade will increase, new
capital will be attracted, the price of pig-iron will fall, but the demand for
it increase; and eventually the greater number of furnaces will more than make
up for the diminished consumption of each.
}

\textbf{Rebound effect}: this idea asserts that gains in energy efficiency are
not always realized because of other systematic effects. For example, savings
from fuel economy improvements can lead to increased driving. Or, more
efficient heating could lead people to keep their homes warmer in the winer.
A similar concept called the Khazzoom-Brookes postulate.

According to a study by EC, air-conditioning has a possible rebound up to
50\%, with an average of 25\%.

\subsection{Theories of Social Change: Ecological Modernication and Social
Movements}

Collaborative approaches:
\begin{itemize}
	\item policy-level interventions
\end{itemize}

Adversarial approaches:
\begin{itemize}
	\item protests
	\item lawsuits
	\item boycotts
\end{itemize}

\subsection{Political ecology}

In political ecology, discourse analysis is used to identify how
environmental problems become naturalized or deemed inevitable.

\subsubsection{A rhetorical model for environmental discourse and its political
discourse}

\textbf{Ethos}: relies on the reputation of the speaker and the appeal to
character, regulatory discourse. Example: \textit{Scientists agree that the
products of biotechnology present no new risks}

\textbf{Pathos}: relies on how emotions are evoked, poetic discourse, nature as
spirit, evokes values. Example: \textit{anti-genetic engineering activists are
causing people to starve in Africa}

\textbf{Logos}: relies on how persuasive a case is made, appeals to fact and
reason. Example: \textit{food security depends on advances in plant breeding.
Biotechnology is the most important new plant breeding tool. We need to
utilize biotechnology to improve food security}

\subsection{Global Production Networks (GPN)}

The GPN framework is used to answer various research questions from
understanding colonialism, patterns of economic development, and global
governance to the socio-ecological transformation of natural resources into
commodities and implications for labor.

In research that uses the concept of GPNs, there is a tendency to focus on the
behavior of multinational actors and institutions.

\textbf{Global value chains}: this concept aims to capture the activities that
give rise to global production systems. Firms construct value over space
through sourcing and contracting arrangements, and this approach aims to
understand how these activities are organized and governed. Value is added
across the supply chain as materials move from raw materials to finished
products.

\textbf{Filieres}: A French concept that seeks to explore the chain of
activities related to the production of raw materials into final export
products. Research on Filieres usually follow the commodity beyond its useful
life, as opposed to other analyses, which may stop at the factory gate/site
of production.

\subsection{Social acceptance of energy systems}

\textit{There have also been several lawsuits in Western US directed
at solar developers by Native Americans over burial grounds and sacred sites
(Mulvaney 2019).}

Pasqualetti 2001:

\begin{itemize}
	\item More participatory approaches yield better results than the
	decide-announce-defend strategy.
	\item \textit{Several studies of the social gap in support of wind
	farms suggest concerns about loss of property value. The first round
	of wind turbine installations can lower home values because of the 
	visual impact, but over time, these prices recover,
	as new wind farms become part of a landscape or location; so research
	even shows wind farms can have a positive effect on the housing price}
\end{itemize}

\subsection{Science and Technology Studies (STS)}

STS explores questions that help us understand energy transitions because of
engagements with expert knowledge production and public participation.

The STS term \textbf{socio-technical imaginaries} is an approach to examining
energy futures. Social norms and values are reflected in or shaped by
scientific knowledge. Any sustainable energy strategy will have to be sensitive
to the context and geographies specific to energy transitions.
\textbf{Reflexivity} is a term used to describe careful reflection and
adaptation in policy design to pay attention to how outcomes of energy
transitions are developing.
