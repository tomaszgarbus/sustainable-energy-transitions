\section{Chapter 1}

\textbf{Energy transitions} are socio-technical processes that reshape the
nature or patterns of use of energy resources and/or technologies.

\textbf{A socio-ecological system} describes human and Earth-system
interactions as dynamic, interconnected, and co-produced by nature and society.

International frameworks to evaluate and manage social and environmental
challenges:

\begin{itemize}
	\item Convention of International Trade in Endangered Sepecies (CITES)
	\item Intergovernmental Panel on Climate Change (IPCC)
	\item Montreal Protocol to protect the ozone layer
	\item Agenda 21
	\item Sustainable Development Goals (SDG)
\end{itemize}

In the 70s, the concept of energy transition mostly referred to providing
energy access to poor communities to increase their quality of life. It was
also used then to refer to the growing need for coal resources in southwest
USA. Today, the phrase \textit{energy transition} refers to the move towards
low-carbon economy.

Holocene - current geological epoch, refers to the last 12000 years. Contenders
for the start of anthropocene epoch include the first major imprints on the
atmosphere from fossil fuels -- emissions of methande and carbon dioxide.
Another candidate is the date that marks the start of atmospheric nuclear
testing in early 50s.

International Energy Agency (IEA) (2010) suggests power demand will increase
from 18 trillion watts in 2020 to somewhere between 25 and 30 trillion watts by
2050. 

Starting in 2015,the world started to build more renewable energy than energy
infrastructure to burn fossil energy.

\textbf{Projections} are trends taken into the future based on some existing
trends or some BAU\footnote{business as usual} scenario.

\textbf{Forecasts} are made by taking these projections and modifying them with
assumptions about the future, such as new technologies or different rates of
change.

\textbf{Primary energy sources} are the natural resources taken from the earth:
coal, "wet" natural gas (wet because it contains water, methane, ethane, and
other gases), petroleum, solar and wind power, uranium, and other direct
sources of energy harvested.

\textbf{Final fuel products} and \textbf{energy carriers} are the energy
sources that directly provide energy services.

Example final fuels: gasoline, "dry" natural gas (dry because it mostly
contains methane), wood for a stove or campfire, hydrogen and electricty.

Example energy carriers: electricity, hydrogen, steam.

\textbf{Corporate Social Responsibility (CSR)} is an approach to sustainability
that focuses on encouraging the private sector through voluntary standards,
industry benchmarks to favour sustainable solutions under the pressure from
investors and social and reputational pressure.

\textbf{Wind, Water and Sunlight (WWS)} strategies focus on replacing current
energy systems with one run solely on electrification and renewables.

\textbf{Hard} and \textbf{soft} paths in energy transitions. Hard path refers
to coal and nuclear power, while soft is led by renewables and appropriate
technologies.

Aces of debates in energy transitions:

\begin{itemize}
	\item apolitical
	\item democratic
	\item command and control
	\item global
	\item centralized
	\item private
	\item clean
	\item political
	\item technocratic
	\item market
	\item local
	\item decentralized
	\item public
	\item renewable
\end{itemize}
