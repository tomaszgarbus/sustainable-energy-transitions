\section{Chapter 7: Low-Carbon Electricity Systems}

\subsection{The electricity grid system}

\begin{itemize}
	\item natural monopolies
	\item jurisdictions grant utilities the right to entire markets
	\item in the past many utilities owned both power plants and systems of
	wires
	\item today there are competitive markets for electricity generation
\end{itemize}

A \textbf{grid} is a collection of components and devices that transmit,
distribute, and deliver electricity, including the transmission wires and
towers, distribution wires and towers, substations, transformers, meters, and
other essential parts, sometimes including the power generators, companies,
and workers.

Grid:
\begin{itemize}
	\item power plants generate alternating current (AC) electricity
	\item transformers step up electricity to higher voltage such as
	\begin{itemize}
		\item 100kV
		\item 230000V
		\item 345000V
		\item 440000V
		\item 500000V
		\item 750000V
	\end{itemize}
	\item moving electricity at lower voltage would require very thick
	wires
	\item voltage is "push" or "pressure" of electricity energy
	\item most (but not all) high-voltage transmission lines deliver
	AC current
	\item unlike DC, AC can have multiple taps
	\item once power reaches a distribution substation, it is stepped down
	to several hundred volts and carried on power poles
	\item most outlets are 110-220V
\end{itemize}

\textbf{Social discount rate} is a measure of how much less valuable money is
to an investor over time.

US electrical grid:
\begin{itemize}
	\item divided into 3 sections
	\begin{itemize}
		\item Western US: Western Interconnection (WECC)
		\item Texas: Electric Reliability Council of Texas (ERCOT)
		\item East, Midwest and South: Eastern Interconnection
	\end{itemize}
\end{itemize}

\subsection{Levelized Cost of Electricity (LCOE)}

\begin{itemize}
	\item standard economic metric representing the cost of electricity
	\item allows for comparing electricity sources with varying capacity
	factors
	\item the main drivers of LCOE calculations are:
	\begin{itemize}
		\item capital costs
		\item fuel costs
		\item annualized operating and maintanance costs
		\item financing
		\item capital recovery factor based on specified discount rate
		and power plant life
	\end{itemize}
\end{itemize}

\textbf{Levelized cost of electricity (LCOE)} is the cost of delivering
electricity over some time period. It is commonly measured as cost per unit
energy (US\$ per kWh).

LCOE framework is also used to describe any avoided social and environmental
costs, social costs of carbon, avoided criteria air pollutant emissions, and
reduced utility regulatory compliance costs.

\textbf{Swanson's effect} suggests that doubling production results in
20\% decrease in costs.

\subsection{Power density}
Power density:
\begin{itemize}
	\item describes how much power can be obtained from an energy source
	per unit land or space.
	\item describes power flux per unit of horizontal surface
	\item some units used to express power density include some amount of
	energy (joule, calorie, btu, kWh) per unit weight (gram, pound, ton)
	or volume (cubic centimeter, cubic decimeter, cubic meter).
	\item helpful to provide a sense of whether specific energy resources
	can be matched to power some area, facility, or place.
	\item nuclear and fossil fuels are the most dense sources.
	\item of the renewables, solar power is the most energy dense,
	sometimes also hydropower.
	\item lower densities have wind and geothermal power.
	\item biofuels have the lowest density.
\end{itemize}
