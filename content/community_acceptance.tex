\section{Community acceptability and the energy transition: a citizens'
perspective}
Breffní Lennon, Niall P. Dunphy, Estibaliz Sanvicente

\subsection{Introduction}

The current shift to RES is different from energy transitions of the past due
to diversity of drivers leading it:

\begin{itemize}
	\item climate change
	\item growing awareness of energy-related inequalities
	\item social and behavioural transformations
	\item questioning historical narratives
	\item challenging accepted understandings of democracy and economics
\end{itemize}

In contrast, past transformations were mostly driven by exploitation of new
energy resource.

Social dimension is of equal importance to that of technology.

This paper explores how local people can contribute to the energy transition
through more meaningful and engaged process.

\subsection{Background}

\textit{Energy democracy} movement emphasises justice and equality being
integral to the current energy transition.

How can local communities become empowered to drive project development and
meaningfully engage in the low-carbon energy transitions?

\textit{Information gaps, information deficits} is a term describing the
assumption that by simply providing appropriate amounts of information,
citizens will respond accordingly and refine their behaviours.

Strenghtening democratic legitimacy can help increase social acceptability of
RES adoption.

The paper presents perspectives of citizens from 6 communities in France,
Ireland, Italy, Spain and UK.

Historically, cooperative movement has a track record of helping local
communities access the start-up capital needed to create businesses and jobs.

\textbf{Energy democracy} narratives offer an alternative to neoliberal modes
of capitalism and challenge the assumption of unlimited growth.

Scholars call for:
\begin{itemize}
	\item more nuanced approaches that move beyond the accept/reject
	dichotomy
	\item adherence to three overarching principles of effectiveness,
	efficiency and legitimacy
	\item establishment of new change alliances
	\item overseeing a fairer distribution of the benefits of change
	\item using market mechanisms to facilitate all stakeholders in the
	transition
\end{itemize}

\subsection{RES configurations for social/community acceptability}

Most commonly cited motivations against renewable energy deployment:
\begin{itemize}
	\item high local costs compared to perceived local benefits
	\item inappropriate scale of development
	\item limited citizen involvement in local energy planning
	\item additionally, for large-scale projects such as wind-farms:
	\begin{itemize}
		\item detrimental effects to human health
		\item biodiversity loss
		\item landscape degradation
		\item negative impacts on tourism and property prices
	\end{itemize}
\end{itemize}

Opposition is often attributed to NIMBYism through oversimplified and perhaps
lazy analyses.

When people know little about the technology, acceptance may mostly depend on
trust in actors, as demonstrated by the case of gasifier project in Devon
(p.4).

The participants of the study wanted a far greater say in shaping the
transition to a low-carbon economy, but the opportunities for a meaningful
engagement were low.

Perceived unfairness can result in protests, damanged relationships and divided
communities and that perceptions of unfairness are exacerbated when winners and
losers are created within the community (cited from Gross[37]).

Deployment of renewable energy projects requires business models which:
\begin{itemize}
	\item deliver sufficient financial return
	\item minimize and mitigate impacts
	\item provide for equitable distribution of financial and other
	benefits among (affected) community members
\end{itemize}

\textbf{Satisficing} is a term mixing words \textit{satisfy} and
\textit{suffice} and denotes meeting an acceptability threshold without
necessariliy maximizing any specific objective.

\subsection{Research design and methodology}

\textbf{Citizen jury} is a deliberative democracy technique that is being used
to engage citizens on a range of research topics including health care. Also
known as citizens' assembly.

\subsection{PBM 1: an energy purchasing cooperative}

A group of local people established a cooperative with open membership to all
residents and microbusinesses in the area. Ownership is vested directly in the
coop's members, each contributing equally to the capital through membership
subscription.

The cooperative deals directly with the energy supply company (ESCO). It also
helps the more financially vulnerable members pay their costs.

This approach involves a minimal number of key stakeholders.

\subsection{PBM 2: a commercial wind farm project}

Members of a local community approach a commercial wind company to develop a
wind farm in their area.

They establish Community Development Association (CDA). Members of the CDA
set up a local company to oversee the planning application and any 
community-oriented incentive schemes, a common practice in a number of EU
member states. The new company is a subsidiary of the commercial wind company.

Electricity produced from the wind farm is fed directly into the national grid
at a fixed rate, under the national Feed-In-Tariff (FIT) scheme.

CDA sub-committee and landowners (hosting the wind turbines) end up having the
most local control in this type of project.

Most participants see the majority of the profits from this type of project
leaving the local area to be accumulated by individuals at national or
international level.

\subsection{PBM 3: a locally owned (hydropower) renewable energy project}

A community cooperative established a subsidiary company to develop a
hydro-electric scheme on a local water-course which flows into a
designated national park.

The key goal is to provide community members with electricity at reduced rates.

A portion of the annual income is put into projects that benefit the wider
community, such as free home insulation and zero-interest loans.

The scheme generates enough electricity annually to power over 300 homes with
a projected income of several million euros over a projected 20-year period.

This project scored highly on all measures except extra-local wealth-generating
capacity.

Potential shortcomings:
\begin{itemize}
	\item geographically specific
	\item technically challenging
	\item reliance on government support
\end{itemize}

\subsection{PBM 4: a farmer-owned biogas cooperative partnered with a
district heating cooperative}

Two groups come together:

\begin{itemize}
	\item the first one wants to turn animal waste into biogas production
	\item the second wants to install a combined heat and power (CHP)
	plant to generate electricity and provide district heating to local
	residents
\end{itemize}

In the beginning the two groups establish separate cooperatives. The first uses
pig slurry to produce methange gas. This biogas is then received by the second
coop that runs the CHP plant supplying heat to local consumers.

Caveats with regard to community participation:
\begin{itemize}
	\item biogas producers face regulatory and social challenges compared
	to the natural gas rivals
	\item local resistance to a biogas facility
\end{itemize}

\subsection{PBM 5: municipalities, universities, schools and hospitals
(MUSH) energy producer}

The mayor of a municipality established a local community-owned project with an
objective of increasing the uptake of renewable energy in the area through the
installation of RES systems on public-owned buildings.

The MUSH coop decided to invest in solar photovoltaic (PV) arrays mounted on
the roofs of two public education institutions and a hospital.

Electricity produced from each PV plant is used onsite with any excess
electricity being fed directly back to the national grid.

Income from the sale of electricity is fed into other projects such as
insulation upgrades in schools and hospitals, decreasing the energy demand of
these buildings by over 40\%.

\subsection{PBM 6: an environmental finance service}

A group of local landowners implements proven nature-based business models that
prioritise the restoration of self-sustaining ecosystems that originally
existed in the area.

With help from an NGO they secure loans from a European investment bank
focusing on nature-based business creation and additional backing from a
European capital financing facility also engaged in this area.

The project involves a series of educational programmes for landowners and
local residents, as well as other technical, financial and promotional supports
that encourage active rewilding of land no longer in active agricultural use.

Besides rewilding efforts, micro-power generation and energy configurations
that have the minimal environmental impact.

The group secures a national grant.
