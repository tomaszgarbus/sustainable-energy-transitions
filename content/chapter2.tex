\section{Chapter 2: Fundamentals of Energy Science}

\subsection{Units and examples}

As a fundamental law of the universe, \textbf{energy is always conserved}.

Energy is a discrete quantity, power is a flow rate quantity.

Energy is the amount of power over time, measured in joules (J)\footnote{
	1 newton is the force that accelerates 1 kg at one meter per square
	second. $1N = \frac{kg \cdot m}{s^2}$.

	1 joule is equal to the amount of work done when the force of one
	newton displaces a mass through a distance of one metre in the
	direction of that force. $1J = N \cdot m = \frac{kg \cdot m^2}{s^2}$
}.

Power is amount of energy released over specific time, like a joule per second
(1 watt)\footnote{
	$W = \frac{J}{s} = \frac{kg \cdot m^2}{s^3}$
}.

TNT equivalent is $4.184 \times 10^9J$. Nuclear explosions are measured in
megatons, officially designated to be 1 million tons of TNT equivalent.

1 Btu (British thermal unit) is the amount of energy contained in a single
match stick, equal to 1055J. 1 quad is 1 quadrillion Btu.

Lightbulbs: one 100-watt incandescent light bulb can be replaced with
8-watt to 12-watt LED bulb to emit same amount of light.

A flat-panel television screen uses the order of magnitude of 100 watts. Phone
charger uses around 10 watts. A hair-dryer or microwave uses up to 1000 watts.
An electic car uses over 7000 watts and electric hot water heater nearly
5000 watts.

\textbf{Power} is the rate flow of energy. It is an amount of instantaneous
energy flow. This means that power has the unit dimension of time. A watt is
also described as 1 joule per second.

Horsepower: 1 hp = 745.7 watts.

\subsection{Methane}

Hydrocarbons such as methane (CH$_4$) and propane (C$_3$H$_8$) when reacted
with oxygen (O$_2$) produce carbon dioxide (CO$_2$) and water (H$_2$O).
Methane can be produced by fossil or biogenic sources. Methane in the form of
dry natural gas is used to heat homes and power fuel cells. It is generated
with a mix of fossil natural gas and biogenic sources in landfills due to the
decomposition of organic materials. Similarly, wastewater treatment facilities,
dairies, and other animal food systems can be sources of methane generation.
Because methane is a potent GHG, each molecule releases about 25 times more
GHG pollution, so many mitigation strategies aim to convert CH$_4$ to CO$_2$.

Stoichiometrically, methane combustion produces carbon dioxide, water, and
heat.

CH$_4$ $+$ 2O$_2$ $\rightarrow$ CO$_2$ $+$ 2H$_2$O + heat

\subsection{Ethanol}

Combustion of ethanol is an exothermic reaction and yields heat, water and
CO$_2$.

C$_2$H$_5$OH $\rightarrow$ 2CO$_2$ $+$ 3H$_2$O $+$ heat

Ethanol is a biofuel produced mainly of corn and sugarcane, often blended with
gasoline.

\subsection{Eletricity}

The force of charge separation is represented as
$F = \frac{kqQ}{r^2}$
where:

\begin{itemize}
	\item $k$ is the electric constant
	\item $q$ and $Q$ are charges of the two objects
	\item $r$ is the distance between the objects
\end{itemize}

\textit{
The key relationship to remember with electricity is that electric fields cause
magnetic fields, and vice versa: magnetic fields produce electric currents.
These magnetic and electric fields are present in fluxes that operate
perpendicular to each other and in a direction according to the right-hand
rule. In other words, the rotation of the magnetic field’s flow of force moves
counterclockwise to the direction of electric current flow.
}

Voltage measure electrical potential.

$1V = \frac{1J}{1C}$

Electric current is the flow of electricity through a conductor. Electric
current produces a magnetic field perpendicular to the direction the current
is moving in loops that travel according to the right-hand rule.

\textbf{Ampere's law}: electric fields produce magnetic fields and magnetic
fields produce electric fields.

\subsection{Laws of thermodynamics}

\textbf{Entropy} is the measure of the amount of energy no longer capable of
conversion into work.

\textbf{The first law of thermodynamics} -- the energy law -- states that the
total content of the universe is contant.

\textbf{The second law of thermodynamics} -- the entropy law -- states that the
total entropy is increasing.

\subsection{Exercises}

\subsubsection{Convert 210 kWh into joules}
$756 \cdot 10^6$J

\subsubsection{How many kWh are there in 101000Btu?}
$29.5986$

\subsubsection{Convert 150 kilocalories into giga-joules (GJ)}
$150kcal = 150k \cdot 4184J = 0.0006276 GJ$

\subsubsection{Ten gallons of gasoline contains how much energy (MJ)?}
10 gallons gasoline $\times \frac{121.3MJ}{\text{1 gallon gasoline}} =
1213MJ$

\subsubsection{The US used approximately 102 quads of energy in 2018. Convert
this to (a) TWh (terawatt-hours) and (b) EJ (exajoules)}
Quadrillon = $10^{15}$

$1 kWh = 3600000J$

(a) $102 \text{quads} \times \frac{293 TWh}{1 \text{quad}} = 29886 TWh$

Exajoule = $10^{18}J$

(b) $102 \cdot 10^{15} \cdot 1055J = 107.61 EJ$

\subsubsection{A household uses about 6721 kWh per year. What is the annual
energy consumption of an average household expressed in (a) tons of coal
equivalent (tce)? (b) ton of oil equivalent (toe)?}

(a) $6721 kWh \times \frac{1 \text{tce}}{8141 \text{kWh}} = 0.83\text{tce}$

(b) $0.57 \text{toe}$

\textbf{Exergy} refers to the amount of useful energy available to do work
relative to the system. While energy cannot be created or destroyed, exergy can
be destroyed because it is a measure of energy's potential and degradation.

\subsection{Photon science}

\textit{
For 4.4 billion years, our sun has generated photons as it balances the
pressure of its weight from gravity against the outward push of energy release
from the fusion of hydrogen. The tremendous weight of the sun’s gravity
momentarily produces a hydrogen isotope with an extra proton—two overall. When
the atom relaxes back to the more common hydrogen with only one proton, it
releases the energy in the form of light. The loss of the temporary proton
accelerates a charge, which is where electromagnetic radiation from our sun
originates.
}

The power from the sun is $3 \times 10^{26}$ watts or $1.360W$ per square
meter. Aout $10^{17}$ watts is used by humans on Earth.

The energy of a photon is represented as (E). Plank's constant (f) and the
speed of light (c) are variables in the determination of energy, which ranges
depending on the wavelight ($\lambda$) of light.
$E = hf = \frac{hc}{\lambda}$

\subsection{GHG}

The ratio of carbon to hydrogen in fuels affects the amount of GHG emmissions
associated with each unit of energy from fossil fuels. For methane, this ratio
is the at 1:4. This is why natural gas emits fewer GHGs per unit energy
than coal where the ratio of carbon to hydrogen is about 2:1, depending on the
grade of coal. Most petroleum has a aratio of about 1:2.
