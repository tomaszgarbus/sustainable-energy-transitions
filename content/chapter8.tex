\section{Chapter 8: Low-carbon mobility}

\subsection{Transportation in 2020 is powered mostly by petroleum}

\begin{itemize}
	\item petroleum is easy to store and transport
	\item fuels from petroleum have very high energy density
	\item they include diesel, petrol (gasoline in US), kerosene, jet fuel,
	liqueified petroleum gas
\end{itemize}

The future of mobility will be driven by new technologies that will facilitate
\begin{enumerate}
	\item automation of driving and transport systems through advances in
	artificial intelligence;
	\item electrification of automobility, which will make the translation
	of energy into wheel motion more efficient;
	\item decarbonization of transport by using more people-powered
	transport (walking and biking) and less carbon-intensive fuels, such
	as solar-powered electricity for EVs, hydrogen generated from clean
	energy, and biofuels.
\end{enumerate}

Why US consumes excessive amounts of oil products:
\begin{itemize}
	\item inefficient fuel fleet
	\item absence of more efficient diesel cars
	\item complete absence of high-velocity trains
\end{itemize}

Paths to decarbonizing transport focus on:

\begin{itemize}
	\item driving less, more biking, avoided travel, walking
	\item replace combustion engines with electric vehicles
	\item decarbonize energy sources that power vehicles
\end{itemize}

\subsection{Electric vehicles}
\begin{itemize}
	\item more than 90\% of criteria air pollutants are created from
	combustion
	\item the full life-cycle economic cost of a vehicle is its cost plus
	the cost of fueling it over some determined time
	\item li-ion batteries are used in all sorts of electronics so EVs
	benefit from the attention of all industries
	\item these batteries do not contain a lot of lithium as a percentage
	of the total battery
	\item Tesla, Nissan and GM use lithium manganese oxide batteries
	\item Toyota uses lithium nickel cobalt oxide
	\item BYD uses lithium iron phosphate
\end{itemize}

Challenges for EVs:
\begin{itemize}
	\item charging stations -- finding places for them in the cities or
	homes with driveways
	\item range anxiety
\end{itemize}

\subsection{Well-to-wheel analysis}

\begin{itemize}
	\item LCA is useful to compare emissions from vehicles and fuels in the
	transportation sector
	\item well-to-wheel (WTW) analysis is specific LCA framework to
	understanding environmental impacts of fuel and vehicle combustion
	\item the findings of this kind of research usually show that the
	fuel yields the highest impact in vehicle's lifeycle
	\item for comparison of different fuels, well-to-tank (WTT) is usually
	used
\end{itemize}

\subsection{Hydrogen fuel cells}

\begin{itemize}
	\item hydrogen is the most common element in the universe (about 75\%)
	\item hydrogen gas is rare on Earth, usually bound to other chemicals
	\item very energy dense
	\item very volatile and risks explosion
	\item \textbf{green hydrogen} typically refers to the hydrogen
	produced by hydrolysis powered by renewable energy\footnote{
		Hydrolysis is a chemical reaction in which a molecule of water
		breaks or more chemical bonds}
	\item hydrogen is an energy carrier, but not primary energy source
	\item \textbf{hydrogen fuel cell} -- a device that converts hydrogen
	and oxygen into water and produces electricity
	\item most hydrogen today is from steam reforming from natural gas
\end{itemize}

\subsection{Ethanol}
\begin{itemize}
	\item most biofuel supply is sugar- and starch-based platforms for
	liquid biofuels
	\item almost one third of US corn produce went to ethanol production in
	2016
	\item \textbf{energy content} of the selected liquid fuels:
	\begin{itemize}
		\item gasoline: 125k Btu/gallon
		\item ethanol: 84k Btu/gallon
		\item compressed natural gas: 106 k Btu/gallon
		\item propane: 91k Btu/gallon
	\end{itemize}
	\item low EROI
	\item food vs fuel debates
	\item water quality and quantity issues are large impacts from
	producing biofuels
	\item nitrogen pollution since farmers overuse cheap fertilizers to
	maximize crops
\end{itemize}

\subsection{Biodiesel and renewable diesel}
\begin{itemize}
	\item different liquid fuels
\end{itemize}

\subsection{Low-carbon drop-in fuels}

\subsection{Vehicle-to-grid storage}

\subsection{Autonomous vehicles}

\subsection{Public transportation}

\subsection{Urban planning for walking and biking}

\subsection{Decarbonizing aviation, long-range travel, flying less}
