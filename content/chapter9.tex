\section{Chapter 9: Low carbon industries and the built environment}

\subsection{Energy efficiency and green building}
\begin{itemize}
	\item nearly 40\% of natural resources extracted from the Earth end up
	in buildings
	\item industries use about a third of overall energy
	\item housing affordability crisis in many parts of the world
	\item greening up the city sometimes results in gentrification: pushing
	poor communities out for the sake of middle class
	\item heating, ventilation and air-conditioning (HVAC) systems use over
	50\% of buildings' energy and 20\% total consumption in US
	\item closing the loop on material flows and transitioning to a
	circular economy could have major implications for natural resource
	use
	\item a \textbf{negawatt} is a term coined by Amory Lovins which
	describes an avoided amount of power
	\begin{itemize}
		\item when a 100-watt lightbulb is replaced with the same
		lumen LED but at 20 watts, it results in 80 negawatts
	\end{itemize}
	\item one of recent improvements is in heat pumps for residential
	heating applications
	\begin{itemize}
		\item some heat pumps are air-to-air, finding heat in the
		outdoors on a cold day and delivering it indoors
		\item geothermal heat pumps are ground-to-air, meaning they
		find warm air in the ground and deliver it indoors
		\item heat pumps can also heat water (air-to-water heat pumps,
		they take heat from the outside and deliver it to water)
		\item air-to-water heat pump can be used to heat a home or
		building with hydronic or radiant floors
		\item as heat pumps replace oil and gas for heat applications,
		the sector will incur GHG benefits and savings
	\end{itemize}
	\item challenge with greening the buildings: turnover of inventory is
	very slow
	\item \textbf{The Rosenfeld Effect}
	\begin{itemize}
		\item total per capita electricity use has stayed relatively
		flat in California in the last 4 decades while it has risen
		sharply in US as a whole
		\item this is often credited to physicist Art Rosenfeld's
		influence on California energy policy
		\item Rosenfeld started championing energy efficiency in the
		early 1970s as a cost-effective strategy to save energy
		resources and reduce customer energy bills
	\end{itemize}
\end{itemize}

\subsection{Water and wastewater infrastructure}
\begin{itemize}
	\item a major energy use and source of GHGs in any municipality or
	community is related to the provision of drinking water and the
	disposal of wastewater
	\item GHGs wastewater treatment is equal to those from aviation and
	container shipping (all 3 represent about 1\% of total emissions)
	\item global sanitation is a major challenge with billions lacking
	access to basic sanitation and clean drinking water
	\item wastewater emissions can contain the GHG methane as well
	\item this is because anaerobic decomposition of organic matter takes
	place in sewers, also called as methanogenesis
	\item since methane is more potent GHG than CO$_2$, it is an
	opportunity to mititage some GHG emissions while generating heat or
	electricity on-site
	\item \textbf{methanogenesis} is a metabolic pathway where
	microorganisms anaerobically digest organic material and produce
	methane
\end{itemize}

\subsection{Cement production}
\begin{itemize}
	\item cement is the second most substance on Earth after water
	\item the main input for cement is limestone, along with calcium and
	silicon
	\item supply chains for cement begin at limestone quarries, in
	addition to sand
	\item the technique most commonly used to produce cement in kilns is
	not only energy intensive but also emits CO$_2$ by chemical reaction
	\item each ton of cement results in about a ton of CO$_2$
	\item cement is estimated to account for 6\% of overall GHG emissions
	\item the following stoichiometry produces CO$_2$ emissions:
	\item CaCO$_2$ + Heat $\rightarrow$ CaO + CO$_2$
	\item short-term changes to cement productions that could reduce
	emissions:
	\begin{itemize}
		\item fuel switching to less carbon-intensive fuels
		\item material substitution: calcium sulfoaluminates and
		calcium silicates are two most critical materials to reduce the
		emission-causing clinker materials
		\item other air pollutants from cement production include
		persistent organic pollutants, dioxins, heavy metals, sulfur
		dioxide, and particulate matter
	\end{itemize}
	\item growth of cement is still slated to continue to rise in the
	coming decades
	\item cement kiln dust is a widely accepted occupational hazard
\end{itemize}

\subsection{Major alloys and metals: steel, copper, aluminium}
\begin{itemize}
	\item mining industries and their associated smelting activities are
	significant contributors to many environmental problems including:
	\begin{itemize}
		\item GHGs
		\item water and air quality impairing emissions
		\item land-use change
	\end{itemize}
	\item iron- and steel-making are about 5\% of global GHGs
	\item every ton of steel results in 1.6-3.1 tons of CO$_2$ equivalent
	\item this is due to coking process
	\item some fundamental processes cannot change
	\item but some companies are mostly relying on renewable electricity
	purchased off-site, coupled with wind and hydropower
	\item circular economy approaches:
	\begin{itemize}
		\item recycled materials for metals can reduce the
		environmental burden
		\item closing this loop decreases energy and GHGs compared to
		virgin metals, mostly steel, copper, aluminium
		\item semiconductor metal recovery reduces environmental
		burdens of PVs
		\item recycling metals is getting more complicated as
		different kinds of metals are brought together where
		historically materials were homogenous
		\item copper contamination from electric components in cars
		is complicating steel recycling
		\item increased plastic composites also complicate this
		picture
		\item recycled aluminium can reduce energy use by almost
		tenfold compared to making it from bauxite
	\end{itemize}
\end{itemize}

\subsection{Chemical industries}
\begin{itemize}
	\item chemical industry is responsible for 10\% of global GHGs
	emissions
	\item about two-thirds of overall energy use in indudstry is used by
	manufacturers using high temperature heat such as in making plastics
	and ammonia
	\item many chemical industries rely on steam and super-heated steam
	for heating reactor vessels and other tasks
	\item ammonia could also be used as a fuel directly, or cracked, for
	its hydrogen
	\item many feedstocks used in chemical industries are purchased from
	oil and gas industries
	\item chemical industries also often buy raw materials from petroleum
	companies
	\begin{itemize}
		\item one example is ethane, which is separated by a cracker
	\end{itemize}
\end{itemize}
